\documentclass{article}
\usepackage{fullpage,amssymb,amsmath,epsf}
\usepackage[12pt]{extsizes}
\usepackage{mathtools}
\usepackage{graphicx}
\usepackage{epstopdf}
\numberwithin{equation}{section}

\setlength{\textheight}{8in}

\usepackage[hang,flushmargin]{footmisc}

\usepackage{url}

\newcommand\blfootnote[1]{%
	\begingroup
	\renewcommand\thefootnote{}\footnote{#1}%
	\addtocounter{footnote}{-1}%
	\endgroup
}

\pagestyle{myheadings}


\documentclass{article}
\usepackage[top = 1.0in]{geometry}

\usepackage{graphicx}

\usepackage[utf8]{inputenc}
\usepackage{listings}
\usepackage[dvipsnames]{xcolor}
\usepackage{bm}
\usepackage{algorithm}
\usepackage{algpseudocode}
\usepackage{framed}

\definecolor{codegreen}{rgb}{0,0.6,0}
\definecolor{codegray}{rgb}{0.5,0.5,0.5}
\definecolor{codepurple}{rgb}{0.58,0,0.82}
\definecolor{backcolour}{rgb}{0.95,0.95,0.92}

\lstdefinestyle{mystyle}{
    backgroundcolor=\color{backcolour},   
    commentstyle=\color{codegreen},
    keywordstyle=\color{magenta},
    stringstyle=\color{codepurple},
    basicstyle=\ttfamily\footnotesize,
    breakatwhitespace=false,         
    breaklines=true,                 
    captionpos=b,                    
    keepspaces=true,                 
    numbersep=5pt,                  
    showspaces=false,                
    showstringspaces=false,
    showtabs=false,                  
    tabsize=2
}

\lstset{style=mystyle}

\newcommand{\di}{{d}}
\newcommand{\nexp}{{n}}
\newcommand{\nf}{{p}}
\newcommand{\vcd}{{\textbf{D}}}
\newcommand{\Int}{\mathbb{Z}}
\newcommand\bb{\ensuremath{\mathbf{b}}}
\newcommand\bs{\ensuremath{\mathbf{s}}}
\newcommand\bp{\ensuremath{\mathbf{p}}}
\newcommand{\relu} { \operatorname{ReLU} }
\newcommand{\smx} { \operatorname{softmax} }
\newcommand\bx{\ensuremath{\mathbf{x}}}
\newcommand\bh{\ensuremath{\mathbf{h}}}
\newcommand\bc{\ensuremath{\mathbf{c}}}
\newcommand\bW{\ensuremath{\mathbf{W}}}
\newcommand\by{\ensuremath{\mathbf{y}}}
\newcommand\bo{\ensuremath{\mathbf{o}}}
\newcommand\be{\ensuremath{\mathbf{e}}}
\newcommand\ba{\ensuremath{\mathbf{a}}}
\newcommand\bu{\ensuremath{\mathbf{u}}}
\newcommand\bv{\ensuremath{\mathbf{v}}}
\newcommand\bP{\ensuremath{\mathbf{P}}}
\newcommand\bg{\ensuremath{\mathbf{g}}}
\newcommand\bX{\ensuremath{\mathbf{X}}}
% real numbers R symbol
\newcommand{\Real}{\mathbb{R}}

% encoder hidden
\newcommand{\henc}{\bh^{\text{enc}}}
\newcommand{\hencfw}[1]{\overrightarrow{\henc_{#1}}}
\newcommand{\hencbw}[1]{\overleftarrow{\henc_{#1}}}

% encoder cell
\newcommand{\cenc}{\bc^{\text{enc}}}
\newcommand{\cencfw}[1]{\overrightarrow{\cenc_{#1}}}
\newcommand{\cencbw}[1]{\overleftarrow{\cenc_{#1}}}

% decoder hidden
\newcommand{\hdec}{\bh^{\text{dec}}}

% decoder cell
\newcommand{\cdec}{\bc^{\text{dec}}}

\usepackage[hyperfootnotes=false]{hyperref}
\hypersetup{
  colorlinks=true,
  linkcolor = blue,
  urlcolor  = blue,
  citecolor = blue,
  anchorcolor = blue,
  pdfborderstyle={/S/U/W 1}
}
\usepackage{nccmath}
\usepackage{mathtools}
\usepackage{graphicx,caption}
\usepackage[shortlabels]{enumitem}
\usepackage{epstopdf,subcaption}
\usepackage{psfrag}
\usepackage{amsmath,amssymb,epsf}
\usepackage{verbatim}
\usepackage{cancel}
\usepackage{color,soul}
\usepackage{bbm}
\usepackage{listings}
\usepackage{setspace}
\usepackage{float}
\definecolor{Code}{rgb}{0,0,0}
\definecolor{Decorators}{rgb}{0.5,0.5,0.5}
\definecolor{Numbers}{rgb}{0.5,0,0}
\definecolor{MatchingBrackets}{rgb}{0.25,0.5,0.5}
\definecolor{Keywords}{rgb}{0,0,1}
\definecolor{self}{rgb}{0,0,0}
\definecolor{Strings}{rgb}{0,0.63,0}
\definecolor{Comments}{rgb}{0,0.63,1}
\definecolor{Backquotes}{rgb}{0,0,0}
\definecolor{Classname}{rgb}{0,0,0}
\definecolor{FunctionName}{rgb}{0,0,0}
\definecolor{Operators}{rgb}{0,0,0}
\definecolor{Background}{rgb}{0.98,0.98,0.98}
\lstdefinelanguage{Python}{
    numbers=left,
    numberstyle=\footnotesize,
    numbersep=1em,
    xleftmargin=1em,
    framextopmargin=2em,
    framexbottommargin=2em,
    showspaces=false,
    showtabs=false,
    showstringspaces=false,
    frame=l,
    tabsize=4,
    % Basic
    basicstyle=\ttfamily\footnotesize\setstretch{1},
    backgroundcolor=\color{Background},
    % Comments
    commentstyle=\color{Comments}\slshape,
    % Strings
    stringstyle=\color{Strings},
    morecomment=[s][\color{Strings}]{"""}{"""},
    morecomment=[s][\color{Strings}]{'''}{'''},
    % keywords
    morekeywords={import,from,class,def,for,while,if,is,in,elif,else,not,and,or
    ,print,break,continue,return,True,False,None,access,as,,del,except,exec
    ,finally,global,import,lambda,pass,print,raise,try,assert},
    keywordstyle={\color{Keywords}\bfseries},
    % additional keywords
    morekeywords={[2]@invariant},
    keywordstyle={[2]\color{Decorators}\slshape},
    emph={self},
    emphstyle={\color{self}\slshape},
%
}
\lstMakeShortInline|

\pagestyle{empty} \addtolength{\textwidth}{1.0in}
\addtolength{\textheight}{0.5in}
\addtolength{\oddsidemargin}{-0.5in}
\addtolength{\evensidemargin}{-0.5in}
\newcommand{\ruleskip}{\bigskip\hrule\bigskip}
\newcommand{\nodify}[1]{{\sc #1}}
\newenvironment{answer}{\sf \begingroup\color{ForestGreen}}{\endgroup}%

\setlist[itemize]{itemsep=2pt, topsep=0pt}
\setlist[enumerate]{itemsep=6pt, topsep=0pt}

\setlength{\parindent}{0pt}
\setlength{\parskip}{4pt}
\setlist[enumerate]{parsep=4pt}
\setlength{\unitlength}{1cm}

\renewcommand{\Re}{{\mathbb R}}
\newcommand{\R}{\mathbb{R}}
\newcommand{\what}[1]{\widehat{#1}}

\renewcommand{\comment}[1]{}
\newcommand{\mc}[1]{\mathcal{#1}}
\newcommand{\half}{\frac{1}{2}}

\DeclareMathOperator*{\argmin}{arg\,min}

\def\KL{D_{KL}}
\def\xsi{x^{(i)}}
\def\ysi{y^{(i)}}
\def\zsi{z^{(i)}}
\def\E{\mathbb{E}}
\def\calN{\mathcal{N}}
\def\calD{\mathcal{D}}
\def\slack{\url{http://xcs229-scpd.slack.com/}}
\def\zipscriptalt{\texttt{python zip\_submission.py}}
\DeclarePairedDelimiter\abs{\lvert}{\rvert}%
 
\usepackage{bbding}
\usepackage{pifont}
\usepackage{wasysym}
\usepackage{amssymb}
\usepackage{framed}
\usepackage{scrextend}

\newcommand{\alns}[1] {
	\begin{align*} #1 \end{align*}
}

\newcommand{\pd}[2] {
 \frac{\partial #1}{\partial #2}
}
\renewcommand{\Re} { \mathbb{R} }
\newcommand{\btx} { \mathbf{\tilde{x}} }
\newcommand{\bth} { \mathbf{\tilde{h}} }
\newcommand{\sigmoid} { \operatorname{\sigma} }
\newcommand{\CE} { \operatorname{CE} }
\newcommand{\byt} { \hat{\by} }
\newcommand{\yt} { \hat{y} }

\newcommand{\oft}[1]{^{(#1)}}
\newcommand{\fone}{\ensuremath{F_1}}

\newcommand{\ac}[1]{ {\color{red} \textbf{AC:} #1} }
\newcommand{\ner}[1]{\textbf{\color{blue} #1}}
\usepackage{graphicx}
\usepackage{subcaption}
%\renewcommand{\epsffile}[1]{
%	\includegraphics[width=\epsfxsize]{#1}
%}

\newcommand{\di}{{d}}
\newcommand{\nexp}{{n}}
\newcommand{\vcd}{{\textbf{D}}}


\usepackage{titling}
\setlength{\droptitle}{-5em}


\begin{document}
\title{XCS229: Additional Notes on Backpropagation}
\date{}
\maketitle

\section{Forward propagation}

Recall that given input $x$, we define $a^{[0]} = x$.
Then for layer $\ell = 1,2,\dots,N$, where $N$ is the number of layers of
the network, we have
\begin{enumerate}
  \item $z^{[\ell]} = W^{[\ell]} a^{[\ell-1]} + b^{[\ell]}$
  \item $a^{[\ell]} = g^{[\ell]}(z^{[\ell]})$
\end{enumerate}
In these notes we assume the nonlinearities $g^{[\ell]}$ are the same for all layers
besides layer~$N$. This is because in the output layer we may be doing
regression [hence we might use $g(x) = x$] or binary classification
[$g(x) = \text{sigmoid}(x)$] or multiclass classification [$g(x) = \text{softmax}(x)$].
Hence we distinguish $g^{[N]}$ from $g$, and assume $g$ is used for all layers besides layer $N$.

Finally, given the output of the network $a^{[N]}$, which we will more simply denote as
$\hat{y}$, we measure the loss
$J(W, b) = \mathcal{L}(a^{[N]}, y) = \mathcal{L}(\hat{y}, y)$. For example, for real-valued regression we might use
the squared loss
$$\mathcal{L}(\hat{y}, y) = \frac{1}{2}(\hat{y} - y)^2$$
and for binary classification using logistic regression we use
$$\mathcal{L}(\hat{y}, y) = -(y \log \hat{y} + (1-y) \log(1-\hat{y}))$$
or negative log-likelihood. Finally, for softmax regression over $k$ classes, we use
the cross entropy loss
$$\mathcal{L}(\hat{y}, y) = -\sum_{j=1}^k \mathbf{1}\{y=j\} \log \hat{y}_j$$
which is simply negative log-likelihood extended to the multiclass setting.
Note that $\hat{y}$ is a $k$-dimensional vector in this case.
If we use $y$ to instead denote the $k$-dimensional vector of zeros with
a single $1$ at the $l$th position, where the true label is $l$, we can also
express the cross entropy loss as
$$\mathcal{L}(\hat{y}, y) = -\sum_{j=1}^k y_j \log \hat{y}_j$$

\section{Backpropagation}

Let's define one more piece of notation that'll be useful for
backpropagation.\footnote{These notes are closely adapted from:\\\url{http://ufldl.stanford.edu/tutorial/supervised/MultiLayerNeuralNetworks/}\\Scribe: Ziang Xie}
We will define
$$\delta^{[\ell]} = \nabla_{z^{[\ell]}} \mathcal{L}(\hat{y}, y)$$

We can then define a three-step  ``recipe" for computing the gradients with respect to
every $W^{[\ell]}, b^{[\ell]}$ as follows:
\begin{enumerate}
  \item For output layer $N$, we have
    $$\delta^{[N]} = \nabla_{z^{[N]}} \mathcal{L}(\hat{y}, y)$$
    Sometimes we may want to compute $\nabla_{z^{[N]}} \mathcal{L}(\hat{y}, y)$ directly
    (e.g. if $g^{[N]}$ is the softmax function),
    whereas other times (e.g. when $g^{[N]}$ is the sigmoid function $\sigma$)
    we can apply the chain rule:
    $$\nabla_{z^{[N]}} \mathcal{L}(\hat{y}, y) = \nabla_{\hat{y}} \mathcal{L}(\hat{y}, y) \circ (g^{[N]})'(z^{[N]})$$
    Note $(g^{[N]})'(z^{[N]})$ denotes the elementwise derivative w.r.t. $z^{[N]}$.
  \item For $\ell = N-1, N-2, \dots, 1$, we have
    $$\delta^{[\ell]} = ({W^{[\ell+1]}}^\top \delta^{[\ell + 1]}) \circ g'(z^{[\ell]})$$
  \item Finally, we can compute the gradients for layer $\ell$ as
    \begin{align*}
      \nabla_{W^{[\ell]}} J(W, b) &= \delta^{[\ell]} {a^{[\ell-1]}}^\top\\
      \nabla_{b^{[\ell]}} J(W, b) &= \delta^{[\ell]}
    \end{align*}
\end{enumerate}
where we use $\circ$ to indicate the elementwise product.
Note the above procedure is for a single training example.

You can try applying the above algorithm to logistic regression
($N=1$, $g^{[1]}$ is the sigmoid function $\sigma$) to sanity check steps (1) and (3).
Recall that $\sigma'(z) = \sigma(z) \circ (1-\sigma(z))$ and $\sigma(z^{[1]})$ is simply $a^{[1]}$.
Note that for logistic regression, if $x$ is a column vector in $\mathbb{R}^{\di\times 1}$, then
$W^{[1]} \in \mathbb{R}^{1 \times \di}$, and hence $\nabla_{W^{[1]}} J(W, b) \in \mathbb{R}^{1 \times \di}$.
Example code for two layers is also given at:
\begin{center}
\url{http://cs229.stanford.edu/notes2020fall/}
\end{center}

% TODO Vectorization.

\end{document}
