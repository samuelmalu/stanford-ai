\documentclass{article}
%\documentstyle[11pt,handout,psfig]{article}

\usepackage{fullpage,amssymb,amsmath,epsf}
\usepackage[12pt]{extsizes}

%These give really tight margins:
%\setlength{\topmargin}{-0.3in}
%\setlength{\textheight}{8.10in}
%\setlength{\textwidth}{5.8in}           
%\setlength{\baselineskip}{0.1875in}     
%\addtolength{\leftmargin}{-2.775in}
%\setlength{\footskip}{0.45in}
%\setlength{\oddsidemargin}{0.5in}
%\setlength{\evensidemargin}{0.5in}
%%\setlength{\headsep}{0pt}
%%\setlength{\headheight}{0pt}

%\setlength{\topmargin}{-0.5in}
\setlength{\textheight}{8in}
%\setlength{\textwidth}{5.0in}           
%\setlength{\baselineskip}{0.1875in}     
%\addtolength{\leftmargin}{-2.775in}
%\setlength{\footskip}{0.45in}
%\setlength{\oddsidemargin}{0.5in}
%\setlength{\evensidemargin}{0.5in}
%%\setlength{\headsep}{0pt}
%%\setlength{\headheight}{0pt}


\markright{XCS229}
\pagestyle{myheadings}


\documentclass{article}
\usepackage[top = 1.0in]{geometry}

\usepackage{graphicx}

\usepackage[utf8]{inputenc}
\usepackage{listings}
\usepackage[dvipsnames]{xcolor}
\usepackage{bm}
\usepackage{algorithm}
\usepackage{algpseudocode}
\usepackage{framed}

\definecolor{codegreen}{rgb}{0,0.6,0}
\definecolor{codegray}{rgb}{0.5,0.5,0.5}
\definecolor{codepurple}{rgb}{0.58,0,0.82}
\definecolor{backcolour}{rgb}{0.95,0.95,0.92}

\lstdefinestyle{mystyle}{
    backgroundcolor=\color{backcolour},   
    commentstyle=\color{codegreen},
    keywordstyle=\color{magenta},
    stringstyle=\color{codepurple},
    basicstyle=\ttfamily\footnotesize,
    breakatwhitespace=false,         
    breaklines=true,                 
    captionpos=b,                    
    keepspaces=true,                 
    numbersep=5pt,                  
    showspaces=false,                
    showstringspaces=false,
    showtabs=false,                  
    tabsize=2
}

\lstset{style=mystyle}

\newcommand{\di}{{d}}
\newcommand{\nexp}{{n}}
\newcommand{\nf}{{p}}
\newcommand{\vcd}{{\textbf{D}}}
\newcommand{\Int}{\mathbb{Z}}
\newcommand\bb{\ensuremath{\mathbf{b}}}
\newcommand\bs{\ensuremath{\mathbf{s}}}
\newcommand\bp{\ensuremath{\mathbf{p}}}
\newcommand{\relu} { \operatorname{ReLU} }
\newcommand{\smx} { \operatorname{softmax} }
\newcommand\bx{\ensuremath{\mathbf{x}}}
\newcommand\bh{\ensuremath{\mathbf{h}}}
\newcommand\bc{\ensuremath{\mathbf{c}}}
\newcommand\bW{\ensuremath{\mathbf{W}}}
\newcommand\by{\ensuremath{\mathbf{y}}}
\newcommand\bo{\ensuremath{\mathbf{o}}}
\newcommand\be{\ensuremath{\mathbf{e}}}
\newcommand\ba{\ensuremath{\mathbf{a}}}
\newcommand\bu{\ensuremath{\mathbf{u}}}
\newcommand\bv{\ensuremath{\mathbf{v}}}
\newcommand\bP{\ensuremath{\mathbf{P}}}
\newcommand\bg{\ensuremath{\mathbf{g}}}
\newcommand\bX{\ensuremath{\mathbf{X}}}
% real numbers R symbol
\newcommand{\Real}{\mathbb{R}}

% encoder hidden
\newcommand{\henc}{\bh^{\text{enc}}}
\newcommand{\hencfw}[1]{\overrightarrow{\henc_{#1}}}
\newcommand{\hencbw}[1]{\overleftarrow{\henc_{#1}}}

% encoder cell
\newcommand{\cenc}{\bc^{\text{enc}}}
\newcommand{\cencfw}[1]{\overrightarrow{\cenc_{#1}}}
\newcommand{\cencbw}[1]{\overleftarrow{\cenc_{#1}}}

% decoder hidden
\newcommand{\hdec}{\bh^{\text{dec}}}

% decoder cell
\newcommand{\cdec}{\bc^{\text{dec}}}

\usepackage[hyperfootnotes=false]{hyperref}
\hypersetup{
  colorlinks=true,
  linkcolor = blue,
  urlcolor  = blue,
  citecolor = blue,
  anchorcolor = blue,
  pdfborderstyle={/S/U/W 1}
}
\usepackage{nccmath}
\usepackage{mathtools}
\usepackage{graphicx,caption}
\usepackage[shortlabels]{enumitem}
\usepackage{epstopdf,subcaption}
\usepackage{psfrag}
\usepackage{amsmath,amssymb,epsf}
\usepackage{verbatim}
\usepackage{cancel}
\usepackage{color,soul}
\usepackage{bbm}
\usepackage{listings}
\usepackage{setspace}
\usepackage{float}
\definecolor{Code}{rgb}{0,0,0}
\definecolor{Decorators}{rgb}{0.5,0.5,0.5}
\definecolor{Numbers}{rgb}{0.5,0,0}
\definecolor{MatchingBrackets}{rgb}{0.25,0.5,0.5}
\definecolor{Keywords}{rgb}{0,0,1}
\definecolor{self}{rgb}{0,0,0}
\definecolor{Strings}{rgb}{0,0.63,0}
\definecolor{Comments}{rgb}{0,0.63,1}
\definecolor{Backquotes}{rgb}{0,0,0}
\definecolor{Classname}{rgb}{0,0,0}
\definecolor{FunctionName}{rgb}{0,0,0}
\definecolor{Operators}{rgb}{0,0,0}
\definecolor{Background}{rgb}{0.98,0.98,0.98}
\lstdefinelanguage{Python}{
    numbers=left,
    numberstyle=\footnotesize,
    numbersep=1em,
    xleftmargin=1em,
    framextopmargin=2em,
    framexbottommargin=2em,
    showspaces=false,
    showtabs=false,
    showstringspaces=false,
    frame=l,
    tabsize=4,
    % Basic
    basicstyle=\ttfamily\footnotesize\setstretch{1},
    backgroundcolor=\color{Background},
    % Comments
    commentstyle=\color{Comments}\slshape,
    % Strings
    stringstyle=\color{Strings},
    morecomment=[s][\color{Strings}]{"""}{"""},
    morecomment=[s][\color{Strings}]{'''}{'''},
    % keywords
    morekeywords={import,from,class,def,for,while,if,is,in,elif,else,not,and,or
    ,print,break,continue,return,True,False,None,access,as,,del,except,exec
    ,finally,global,import,lambda,pass,print,raise,try,assert},
    keywordstyle={\color{Keywords}\bfseries},
    % additional keywords
    morekeywords={[2]@invariant},
    keywordstyle={[2]\color{Decorators}\slshape},
    emph={self},
    emphstyle={\color{self}\slshape},
%
}
\lstMakeShortInline|

\pagestyle{empty} \addtolength{\textwidth}{1.0in}
\addtolength{\textheight}{0.5in}
\addtolength{\oddsidemargin}{-0.5in}
\addtolength{\evensidemargin}{-0.5in}
\newcommand{\ruleskip}{\bigskip\hrule\bigskip}
\newcommand{\nodify}[1]{{\sc #1}}
\newenvironment{answer}{\sf \begingroup\color{ForestGreen}}{\endgroup}%

\setlist[itemize]{itemsep=2pt, topsep=0pt}
\setlist[enumerate]{itemsep=6pt, topsep=0pt}

\setlength{\parindent}{0pt}
\setlength{\parskip}{4pt}
\setlist[enumerate]{parsep=4pt}
\setlength{\unitlength}{1cm}

\renewcommand{\Re}{{\mathbb R}}
\newcommand{\R}{\mathbb{R}}
\newcommand{\what}[1]{\widehat{#1}}

\renewcommand{\comment}[1]{}
\newcommand{\mc}[1]{\mathcal{#1}}
\newcommand{\half}{\frac{1}{2}}

\DeclareMathOperator*{\argmin}{arg\,min}

\def\KL{D_{KL}}
\def\xsi{x^{(i)}}
\def\ysi{y^{(i)}}
\def\zsi{z^{(i)}}
\def\E{\mathbb{E}}
\def\calN{\mathcal{N}}
\def\calD{\mathcal{D}}
\def\slack{\url{http://xcs229-scpd.slack.com/}}
\def\zipscriptalt{\texttt{python zip\_submission.py}}
\DeclarePairedDelimiter\abs{\lvert}{\rvert}%
 
\usepackage{bbding}
\usepackage{pifont}
\usepackage{wasysym}
\usepackage{amssymb}
\usepackage{framed}
\usepackage{scrextend}

\newcommand{\alns}[1] {
	\begin{align*} #1 \end{align*}
}

\newcommand{\pd}[2] {
 \frac{\partial #1}{\partial #2}
}
\renewcommand{\Re} { \mathbb{R} }
\newcommand{\btx} { \mathbf{\tilde{x}} }
\newcommand{\bth} { \mathbf{\tilde{h}} }
\newcommand{\sigmoid} { \operatorname{\sigma} }
\newcommand{\CE} { \operatorname{CE} }
\newcommand{\byt} { \hat{\by} }
\newcommand{\yt} { \hat{y} }

\newcommand{\oft}[1]{^{(#1)}}
\newcommand{\fone}{\ensuremath{F_1}}

\newcommand{\ac}[1]{ {\color{red} \textbf{AC:} #1} }
\newcommand{\ner}[1]{\textbf{\color{blue} #1}}
\begin{document}
\title{XCS229 Lecture Notes}
\author{Andrew Ng}
\date{}
\maketitle


\noindent
%Note: These notes are preliminary, and will be updated after we finish 
%talking about model selection in class.
%\documentstyle[12pt,psfig,epic]{article}
%\documentstyle[12pt,handout,psfig]{article}
%\documentstyle[12pt,mynotes,psfig]{article}

\def\qed {\hfill\hbox{{\vspace{0.01in} \large $\Box$}}}

%\renewcommand{\epsilon}{{\varepsilon}}
%\newcommand{\ehat}{{\hat{\varepsilon}}}
%\newcommand{\hhat}{{\hat{h}}}
%\newcommand{\hstar}{{h^\ast}}
%\newcommand{\muhat}{{\hat{\mu}}}
%\newcommand{\sign}{{\rm sign}}


\section{The perceptron and large margin classifiers}

%We how SVMs, by using kernels, can use infinite-dimensional 
%feature spaces.  Thus, the VC dimension of the hypothesis
%class used by such SVMs is $\infinity$.  Why don't they overfit?
%
%It turns out that, by finding 

In this final set of notes on learning theory, we will introduce
a different model of machine learning.  Specifically, we have so
far been considering {\bf batch learning} settings in which we are
first given a training set to learn with, and our hypothesis $h$ 
is then evaluated on separate test data.  In this set of notes, we will 
consider the {\bf online learning} setting in which the algorithm
has to make predictions continuously even while it's learning.

In this setting, the learning algorithm 
is given a sequence of examples 
$(x^{(1)}, y^{(1)}), (x^{(2)}, y^{(2)}),  \ldots (x^{(m)}, y^{(m)})$ 
in order.  
Specifically, the algorithm first sees $x^{(1)}$ and is asked to predict 
what it thinks $y^{(1)}$ is. 
After making its prediction, the true value of $y^{(1)}$ is revealed 
to the algorithm (and the algorithm may use this information to perform
some learning).  The algorithm is then shown $x^{(2)}$ and
again asked to make a prediction, after which $y^{(2)}$ is revealed, 
and it may again 
perform some more learning.  This proceeds until we reach $(x^{(m)}, y^{(m)})$.  
In the online learning setting, we are interested in the total number of 
errors made by the algorithm during this process.  Thus, it models 
applications in which the algorithm has to make predictions even while it's
still learning. 

We will give a bound on the online learning error of the
perceptron algorithm.  To make our subsequent derivations easier, we will 
use the notational convention of denoting the class labels by
$y= \in \{-1, 1\}$.  

Recall that the perceptron algorithm has 
parameters $\theta \in \Re^{n+1}$, and makes its predictions according to 
\begin{equation}
 h_\theta(x) = g(\theta^Tx)
\end{equation}
where 
\[
g(z) = \left\{\begin{array}{ll}
    1  & \mbox{if~} z \geq 0 \\
    -1  & \mbox{if~} z < 0.
    \end{array} \right.
\]
Also, given a training example $(x,y)$, the perceptron learning
rule updates the parameters as follows.  If $h_\theta(x) = y$, then 
it makes no change to the parameters. 
Otherwise, it performs 
the update\footnote{This looks slightly different from the update rule
we had written down earlier in the quarter 
because here we have changed the labels to be $y \in \{-1,1\}$.  
Also, the learning rate parameter $\alpha$ was dropped.  The only effect of the 
learning rate is to scale all the parameters $\theta$ by some fixed constant,
which does not affect the behavior of the perceptron.}
\[
\theta := \theta + y x.
\]

The following theorem gives a bound on the online learning error of the perceptron 
algorithm, when it is run as an online algorithm that performs an update 
each time it gets an example wrong.  Note that the bound below on the number 
of errors does not have an explicit dependence on the number of
examples $m$ in the sequence, or on the dimension $n$ of the inputs (!).

\bigskip
\noindent 
{\bf Theorem (Block, 1962, and Novikoff, 1962)}.  Let a sequence of
examples 
$(x^{(1)}, y^{(1)}), (x^{(2)}, y^{(2)}),  \ldots (x^{(m)}, y^{(m)})$ be
given. 
Suppose that 
$||x^{(i)}|| \leq D$ for all $i$, and further that there 
exists a unit-length vector $u$
($||u||_2=1$) such that $y^{(i)} \cdot (u^Tx^{(i)}) \geq \gamma$ for all
examples in the sequence (i.e., $u^Tx^{(i)} \geq \gamma$ if $y^{(i)}=1$, 
and $u^Tx^{(i)} \leq -\gamma$ if $y^{(i)}=-1$, so that $u$ separates 
the data with a margin of at least $\gamma$).  Then the total number 
of mistakes that the perceptron algorithm makes on this sequence is at 
most $(D/\gamma)^2$. 

\medskip
\noindent
{\bf Proof.}  
The perceptron updates its weights only on those examples on which it 
makes a mistake.  Let $\theta^{(k)}$ be the weights that were being used 
when it made its $k$-th mistake.  So, $\theta^{(1)} = \vec{0}$ (since 
the weights are initialized to zero), and if the $k$-th mistake was 
on the example $(x^{(i)}, y^{(i)})$, then 
$g( (x^{(i)})^T \theta^{(k)}) \neq y^{(i)}$, 
which implies that 
\begin{equation}
(x^{(i)})^T \theta^{(k)} y^{(i)} \leq 0.  \label{eqn-negsign}
\end{equation}
Also, from the perceptron learning rule, we would have that
$\theta^{(k+1)} = \theta^{(k)} + y^{(i)} x^{(i)}$. 

We then have 
\begin{eqnarray*}
(\theta^{(k+1)})^T u &=& (\theta^{(k)})^T u + y^{(i)} (x^{(i)})^T u \\
&\geq& (\theta^{(k)})^T u + \gamma 
\end{eqnarray*}
%Since $\theta^{(k)^Tu = \vec{0}^Tu = 0$, 
By a straightforward inductive argument,
implies that 
\begin{equation}
(\theta^{(k+1)})^T u \geq k \gamma.\label{eqn-boundnorm}
\end{equation}

Also, we have that 
\begin{eqnarray}
 || \theta^{(k+1)} ||^2 &=& ||\theta^{(k)} + y^{(i)} x^{(i)}||^2  \nonumber \\
 &=& ||\theta^{(k)}||^2  + ||x^{(i)}||^2  + 2 y^{(i)}(x^{(i)})^T \theta^{(i)}\nonumber \\
 &\leq& ||\theta^{(k)}||^2  + ||x^{(i)}||^2  \nonumber \\
 &\leq& ||\theta^{(k)}||^2  + D^2  \label{eqn-boundnormsqr}
\end{eqnarray}
The third step above used Equation~(\ref{eqn-negsign}).
Moreover, again by applying a straightfoward inductive 
argument, we see that~(\ref{eqn-boundnormsqr}) implies 
\begin{equation}
||\theta^{(k+1)} ||^2 \leq k D^2.  \label{eqn-boundnormsqr2} 
\end{equation}

Putting together~(\ref{eqn-boundnorm}) and~(\ref{eqn-boundnormsqr}) 
we find that 
\begin{eqnarray*}
\sqrt{k} D &\geq& ||\theta^{(k+1)}|| \\
&\geq& (\theta^{(k+1)})^T u \\
&\geq& k \gamma. 
\end{eqnarray*}
The second inequality above follows from the fact that $u$ is a unit-length
vector (and $z^Tu = ||z|| \cdot ||u|| \cos\phi \leq ||z|| \cdot ||u||$, where
$\phi$ is the angle between $z$ and $u$).
Our result implies that $k \leq (D/\gamma)^2$.   Hence, if the perceptron made
a $k$-th mistake, then $k \leq (D/\gamma)^2$. \qed




\end{document}


